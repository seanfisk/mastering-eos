% -*- coding: utf-8; -*-

% \RequirePackage[l2tabu,orthodox]{nag}
% Note: Font size doesn't do anything here. Use scale in the beamerposter usepackage below.
\documentclass{beamer}
\mode<presentation>
{
  \usetheme{EOS}
}
\graphicspath{{images/}}

\usepackage{minted}
\newmint{bash}{frame=single}
\newminted{bash}{gobble=10,frame=single}

% Just loading fontspec should cause LuaLaTeX to pick a font that can display most Unicode characters.
\usepackage{fontspec}
\usepackage{tabu}
\usepackage{microtype}
\usepackage[orientation=landscape,size=a1,scale=1.2]{beamerposter}

\newcommand{\mailtohref}[1]{\urlstyle{same}\href{mailto:#1}{\textless\nolinkurl{#1}\textgreater}}
\newcommand{\command}[1]{\textbf{\texttt{#1}}}

\title{Mastering EOS}
\author[Woodring \& Fisk]{Ira Woodring \mailtohref{woodriir@gvsu.edu} \\
  Sean Fisk \mailtohref{fiskse@mail.gvsu.edu}}
\institute[GVSU]{Grand Valley State University}

\begin{document}
\begin{frame}[fragile]{}
  \begin{columns}[T]
    \begin{column}{0.33\textwidth}
      % \vfill
      % \begin{block}{\large Fontsizes}
      %   \centering
      %   {\tiny tiny}\par
      %   {\scriptsize scriptsize}\par
      %   {\footnotesize footnotesize}\par
      %   {\normalsize normalsize}\par
      %   {\large large}\par
      %   {\Large Large}\par
      %   {\LARGE LARGE}\par
      %   {\veryHuge veryHuge}\par
      %   {\VeryHuge VeryHuge}\par
      %   {\VERYHuge VERYHuge}\par
      % \end{block}
      \begin{block}{Inter-EOS Password-less SSH}
        {\scriptsize \inputminted[tabsize=2,frame=single]{bash}{scripts/ssh.bash}}
        For more security, use a passphrase and \texttt{ssh-agent}.
      \end{block}
      \begin{block}{Manipulating the PATH}
        When you type a command, the shell searches for the executable using the \texttt{PATH} environmental variable. \texttt{PATH} should be set to a colon-delimited list of directories containing executables. To view your \texttt{PATH}, type:
        \begin{bashcode}
          echo $PATH
        \end{bashcode}
        For instance, to add a local scripts directory (conventionally called \url{\~/bin}) to your \texttt{PATH}, run:
        \begin{bashcode}
          export PATH=~/bin:$PATH
        \end{bashcode}
        Note that this causes the shell to find your scripts before system executables of the same name.
      \end{block}
      \begin{block}{Multiple terminals}
        As an alternative to opening multiple terminals running \texttt{ssh} or multiple PuTTY windows, terminal multiplexers such as \texttt{screen} and \texttt{tmux} may be used. \texttt{tmux} is recommended. Type \command{tmux} once SSH'd to start it. \\
        \textbf{Default \texttt{tmux} bindings:} \\
        {\newcommand{\key}[1]{\texttt{C-b #1}}
          \textbf{\key{c}} means \command{Control + b}, then \command{c}.
          \begin{tabu} to 0.9\linewidth { X[2] X X[2] X }
            \hline
            New window & \key{c} & Kill window & \key{\&} \\ \hline
            Last window & \key{l} & Jump to window 4 & \key{4} \\ \hline
            Next window & \key{n} & Previous window & \key{p} \\ \hline
            Enter copy mode & \key{[} & Paste buffer & \key{]} \\ \hline
          \end{tabu} \\[0.5em]
          \textbf{Warning:} Detached sessions will be killed periodically. Please do not leave important jobs running.
        }
      \end{block}
      \begin{block}{Getting help on commands}
        To get help on the \texttt{dd} command...
        \begin{itemize}
        \item Brief help: \texttt{man dd}, available for most commands
        \item Detailed help: \texttt{info dd}, available for some commands
        \end{itemize}
      \end{block}
    \end{column}
    \begin{column}{0.33\textwidth}
      \begin{block}{Version control systems}
        Version control systems are used to track changes to a set of files. Their use is considered professional software development practice.
        \begin{itemize}
        \item Distributed Version Control Systems (recommended)
          \begin{itemize}
          \item Git (git) \url{<git-scm.com>}
          \item Mercurial (hg) \url{<mercurial.selenic.com>}
          \end{itemize}
        \item Centralized Version Control Systems
          \begin{itemize}
          \item Subversion (svn) \url{<subversion.apache.org>}
          \item Concurrent Versions System (cvs) \url{<cvs.nongnu.org>}
          \end{itemize}
        \end{itemize}
      \end{block}
      \begin{block}{Installing software to your EOS account}
        Installs version 1.17.0 of aria2, a download utility similar to wget but with more features. This install is typical of many programs using the GNU build system. The option of note is the \texttt{--prefix} option to \texttt{./configure}. You may install to any directory you like, but \url{\~/.local} is a common convention. \\
        {\scriptsize \inputminted[tabsize=2,frame=single]{bash}{scripts/install-aria2.bash}}
      \end{block}
      \begin{block}{Directory navigation}
        \begin{tabu} to 0.9\linewidth { X X }
          \texttt{pushd} \textit{dir} & Switch to \textit{dir} and add the current directory to the directory stack \\ \hline
          \texttt{popd} & Remove the top directory on the directory stack and switch to it \\ \hline
          \texttt{dirs} & Show contents of the directory stack
        \end{tabu}
      \end{block}
      \begin{block}{Using the clipboard}
        \begin{bashcode}
          alias copy='xclip -selection clipboard -in'
          alias paste='xclip -selection clipboard -out'
          echo 'EOS Rocks' | copy
          copy eos.txt
          paste > eos.txt
        \end{bashcode}
      \end{block}
    \end{column}
    \begin{column}{0.33\textwidth}
      \begin{block}{Downloading files}
        Download a copy of the W4 form.
        \begin{bashcode}
          wget 'http://www.irs.gov/pub/irs-pdf/fw4.pdf'
          curl 'http://www.irs.gov/pub/irs-pdf/fw4.pdf' > fw4.pdf
        \end{bashcode}
      \end{block}
      \begin{block}{Interacting with other users}
        \begin{tabu} to 0.9\linewidth { X X }
          \texttt{w} & Show system uptime, who is logged in, and what they are doing \\ \hline
          \texttt{last} & Show login history \\ \hline
          \texttt{write} & Write a message to another user's terminal \\ \hline
          \texttt{mesg [yn]} & Enable or disable \texttt{write} access to your terminal
        \end{tabu}
      \end{block}
            \begin{block}{Software build systems}
        Using a build system offers an alternative to repetitively typing compiler commands and is considered professional practice.
        \begin{itemize}
        \item C/C++
          \begin{itemize}
          \item SCons \url{<scons.org>} (recommended)
          \item CMake \url{<cmake.org>}
          \item make \url{<gnu.org/s/make>}
          \end{itemize}
        \item Java
          \begin{itemize}
          \item Buildr \url{<buildr.apache.org>} (recommended)
          \item SCons \url{<scons.org>}
          \item Maven \url{<maven.apache.org>}
          \item Ant \url{<ant.apache.org>}
          \end{itemize}
        \end{itemize}
      \end{block}
            \begin{block}{Run \texttt{make} faster by using all processors}
        \bash!make --jobs="$(grep '^processor' /proc/cpuinfo | wc --lines)"!
      \end{block}
      \begin{block}{Miscellaneous}
        \begin{itemize}
        \item If a user is already logged on to a machine, press \command{Ctrl-Alt-Backspace} to kill the X server and log them out.
        \item Change your password with \command{passwd}.
        \item Check your quota with \command{quota --human-readable} or \command{quota -s}.
        \end{itemize}
      \end{block}
    \end{column}
  \end{columns}
\end{frame}

\end{document}